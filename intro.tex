\section{Introduction}
\label{s:intro}

Network switches and routers in modern datacenters, enterprises, and
service provider networks are required to perform a variety of tasks
in addition to standard packet forwarding. The set of requirements for
routers has only been increasing with time as network operators have
sought to exercise greater control over performance and security, and
to support an evolving set of network protocols.

%TODO: Consider Alvin's point about global shared state
% being another feature of domino.
Performance and security may be improved using both data-plane and
control-plane mechanisms. This paper focuses on data-plane algorithms
for traffic management running in a switch. These algorithms process
every data packet, transforming the packet and often also some state
maintained in the switch.  Examples of such algorithms include
congestion control with switch feedback~\cite{xcp, rcp, pdq, dctcp},
active queue management~\cite{red,blue,avq,codel,pie},
network measurement~\cite{opensketch, bitmap_george, elephant_george}, and load-balanced routing in the data
plane~\cite{conga}.

Because data-plane algorithms process every packet, an important
requirement is the ability to process packets at the line rate.  As a
result, these algorithms are typically implemented using dedicated
hardware. However, hardware designs are rigid and prevent
reconfigurability in the field making it difficult to implement and
deploy new algorithms without investing in new hardware---a
time-consuming and expensive proposition.

This rigidity affects vendors~\cite{cisco_nexus, dell_force10,
  arista_7050} building network switches with merchant-silicon
chips~\cite{trident, tomahawk, mellanox}, network operators deploying
switches~\cite{google,facebook,vl2}, and researchers developing new
switch algorithms~\cite{xcp, codel, d3, detail, pdq}.
%
%Today, the only way to implement a new data-plane algorithm at line rate
%is to build hardware for it.
To run data-plane algorithms after a switch has been built and
deployed, researchers and companies have attempted to build
programmable routers for many years, starting from efforts on active
networks~\cite{active-nets} to network processors~\cite{npu_survey} to
software routers~\cite{click,routebricks}. All these efforts have
compromised on speed to provide programmability, typically running an
order of magnitude (or worse) slower than hardware line
rates. Unfortunately, such a reduction in performance has meant that
these systems are rarely deployed in production networks, if at all.

Programmable switching chips~\cite{flexpipe, xpliant, rmt}, which are
competitive with state-of-the-art fixed-function
chipsets~\cite{trident, tomahawk, mellanox}, are a recent alternative.
These chips implement a few low-level hardware primitives that can be
configured by software into a processing
pipeline~\cite{xpliant_sdk,xpliant_sdk2,intel_sdk,rmt,p4}. This
approach is attractive because it does not compromise on data rates
and the area overhead is small.

%Programming these chips has
%become more user-friendly over time.  Initially, such chips were
%programmed using proprietary SDKs such as those from
%XPliant~\cite{xpliant_sdk, xpliant_sdk2} and
%Intel~\cite{intel_sdk}. Such SDKs closely resemble the underlying
%chipset, making them inconvenient for network programmers who are more
%familiar with a language like C.

\if 0
P4~\cite{p4, p4spec} is a packet-processing language for switching chips that
raises the level of abstraction relative to these SDKs.  For instance, P4
allows the programmer to specify packet parsing and processing without
restricting the set of protocol formats or the set of actions that can be
executed when matching packet headers in a match-action table. Many common
data-plane tasks that require header rewriting can now be expressed
naturally in P4, such as handling new protocol formats, switching, forwarding,
tunneling, and access control~\cite{dc_p4}.
\fi

P4~\cite{p4, p4spec} is an emerging packet-processing language for
such chips. P4 allows the programmer to specify packet parsing and
processing without restricting the set of protocol formats or the set
of actions that can be executed when matching packet headers in a
match-action table. Data-plane tasks that require header rewriting can
be expressed naturally in P4.
%, such as handling new protocol formats,
%switching, forwarding, tunneling, and access control~\cite{dc_p4}.

By contrast, traffic management algorithms don't rewrite headers, but
instead manipulate internal switch state in a manner unique to each
algorithm. We believe that network programmers would prefer the
convenience of an imperative language such as C that directly captures
the algorithm's intent without shoehorning them into hardware
constructs such as a sequence of match-action tables like P4 requires
them to.  Furthermore, this is predominantly how such algorithms are
expressed in pseudocode~\cite{red, csfq, codel_code, avq, blue}, and
implemented in sofware routers~\cite{click, dpdk, routebricks},
network processors~\cite{packetc, nova}, and at network
endpoints~\cite{qdisc}.

This paper presents \pktlanguage, a new domain-specific language (DSL)
for data-plane algorithms.  \pktlanguage is an imperative language
with C-like syntax. The key abstraction in \pktlanguage is a {\em
  packet transaction} (\S\ref{s:transactions}): a sequential block of
code that runs to completion on each packet before executing on the
next packet. Packet transactions provide a convenient programming
model, because they allows the programmer to focus on the operations
needed for each packet without worrying about other packets that are
concurrently being processed.

% Anirudh->Alvin: I don't like the phrase "to be executed".
% It sounds like a processor, and is a little wierd to me grammatically.
% Technically, we would say to binaries that execute on a family of abs. machines,
% but that would require explaining how we generate the binaries and again gives
% the impression of a processor.
% compiling to \absmachine is my fix to this, and is what we use in the proposal.

The \pktlanguage compiler that transforms packet transactions to a
family of abstract machines called \absmachine~(\S\ref{s:absmachine})
(for Protocol-Independent Switch Architecture). \absmachine
generalizes the Reconfigurable Match-Action Table~\cite{rmt} model and
captures essential features of line-rate programmable
switches~\cite{rmt, xpliant, flexpipe}. In addition, \absmachine
introduces {\em atoms} to represent atomic computations provided
natively by a \absmachine machine, much like
load-link/store-conditional, and packed-multiply-and-add on x86
machines today~\cite{x86_manual}.  Atoms provide hardware support for
packet transactions, similar to how an atomic test-and-set can
implement an atomic increment.

To evaluate \pktlanguage, we express algorithms such as
RCP~\cite{rcp}, CoDel~\cite{codel}, heavy-hitter
detection~\cite{opensketch}, and CONGA~\cite{conga}, as packet
transactions in \pktlanguage~(\S\ref{s:eval}). Expressing these
algorithms involved a straightforward translation of each algorithm's
reference code to \pktlanguage syntax.  The \pktlanguage compiler
guarantees deterministic performance for these algorithms: all packet
transactions that compile run at line rate on a \absmachine machine,
or will be rejected by the compiler.  We use the \pktlanguage compiler
to determine if each algorithm can run at line rate on several
different \absmachine machines that differ in the atoms they provide
(Table~\ref{tab:algos}).

Our results indicate that it is possible to provide a familiar
programming model, resembling DSLs for software routers and NPUs, and
also achieve line-rate performance. These findings help resolve the
concerns raised in recent work~\cite{p4} that expressive languages are
unsuitable for line-rate packet processing.

%Will move this to the conclusion, otherwise it sounds repetitive.
%contrary to concerns raised in~\cite{p4} regarding the
%unsuitability of expressive languages for line-rate packet processing.


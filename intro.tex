\section{Introduction}
\label{s:intro}

Data-plane algorithms~\cite{cestan} are algorithms that are implemented within
a network switch. These algorithms process every data packet that passes
through the switch, transforming the packet and/or state of the switch itself.
Examples of such algorithms include congestion-control that uses feedback from
switches~\cite{xcp, rcp, pdq, dctcp}, active queue management~\cite{codel},
network measurement~\cite{opensketch, bitmap_george, elephant_george}, and
load-balanced routing in the data plane~\cite{conga}.

Because data-plane algorithms process every packet, an important requirement is
the ability to run at the switch's line rate.  As a result, these algorithms
are primarily implemented using dedicated hardware. However, hardware designs
are rigid and inflexible, making it difficult to experiment with new
algorithms.

This rigidity affects network switch vendors that build network
equipment~\cite{cisco_nexus, dell_force10, arista_7050} based on
merchant-silicon switching chips,  network operators that use such chips within
their private networks~\cite{jupiter,amazon,isp}, and researchers developing
new switch algorithms~\cite{xcp, codel, d3, detail, pdq}. In all cases, the
only way to implement a new data-plane algorithm is to expressly build hardware
for it---a time-consuming and resource-intensive process.

Programmable switching chips~\cite{flexpipe, xpliant, rmt} that are
performance-competitive with state of the art fixed-function
chipsets~\cite{trident, tomahawk, mellanox} have emerged as an alternative.
These chips allow designers to express their algorithms using a domain-specific
language such as P4~\cite{p4}, propreitary SDKs such as the XPliant SDK~\cite{xpliant_sdk},
or APIs such as the switch-abstraction interface~\cite{sai}.
%~\cite{http://files.opencompute.org/oc/public.php?service=files&t=10c4eb2695c253e9e2fa58329fd53a82}
% ~\cite{http://www.cavium.com/newsevents-Cavium-XPliant-Switches-and-Microsoft-Azure-Networking-Achieve-SAI-Routing-Interoperability.html}

% I think PX applies only to FPGAs.  packetC, POF, and all its related cousins
% apply to NPUs only (by and large).  Really, the state-of-the art for
% programmble switching chips is only P4 and proprietary APIs.

These approaches bear a close resemblance to the underlying hardware, forcing
programmers to reason about low-level pipeline semantics and hardware
concurrency, and place a high burden on developers who are used to writing
sequential programs in a high-level language (such as C) for software
routers~\cite{click}, and network processors~\cite{packetc, ixp, microenginec,
nova}. As evidence, we point the reader to
\url{https://github.com/anirudhSK/p4-semantics}, which documents instances of
incorrect P4 code caused by confusion in language semantics.

% Anirudh->Alvin: I am getting rid of this because I now have more direct evidence
% that the sequential + parallel mixup doesn't work, which I presented at the P4
% consortium to get them to switch to sequential.
%%Taken together, this mix of sequential and parallel
%%semantics means that the net effect of even extremely simple P4 programs is
%%hard to state crisply.

%%For instance, simple\_router.p4~\cite{simple_router.p4} is a program that only
%%implements longest-prefix-match IP routing. Its effect, at first glance,
%%is a sequence of two statements: to apply an IPv4 longest prefix match, and
%%then to use the match result  to set the destination MAC address. However,
%%hidden beneath this sequential facade is the fact that the longest prefix match
%%actually executes three operations in parallel: decrementing the IP TTL field,
%%setting the egress port, and the next hop IP address in a switch-internal
%%header.
%%

This paper presents a new DSL, \pktlanguage~(\S\ref{s:language}), for
expressing data-plane algorithms. \pktlanguage is a high-level, imperative
language that allows programmers to express data-plane algorithms using {\em
packet transactions}: sequential blocks of code that run to completion on each
packet and in isolation from other packets executing the same code block.
\pktlanguage is heavily constrained by the capabilities of switch targets: for
instance, it forbids all iterative constructs. Constraining \pktlanguage leads
to a simpler compiler for \pktlanguage~(\S\ref{s:compiler}), relative to
compilers~\cite{ixp} for imperative languages that target more flexible devices
such as network processors and software routers.

The \pktlanguage compiler takes a packet transaction written in \pktlanguage
and generates code for an abstract machine,
\absmachine~(\S\ref{ss:abstract_machine}).  \absmachine introduces the notion
of an \textit{atom}: a sequential code block consisting of multiple
packet-processing instructions. These instructions are assumed to execute
atomically: their effect is visible in entirety before the next packet is
processed. Internally, an atom can encapsulate local state that affects the
atom's behavior from packet to packet, e.g.  switch counters.

Using atoms as a building block, \absmachine specifies a switch pipeline as a
grid of atoms with the horizontal axis representing physical pipeline stages
and the vertical axis representing concurrency within each stage.  By limiting
the number and type of instructions within each atom~(\S\ref{ss:complexity}),
\absmachine models limits on computation in line-rate switches.
% Btw, our limits can be a little more "non-linear" than the number and type of
% instructions, such as a circuit template.

While existing uses of transactions in databases~\cite{db_trans}, network
processors ~\cite{npus}, and software routers~\cite{click} guarantee isolation,
their performance depends on transaction complexity. Packet transactions, by
contrast, are deterministic. All packet transactions that are implementable
either run at the switch's line rate, or are rejected by the compiler.

Expressing data-plane algorithms as packet transactions has an important
practical benefit.  Conceptually, a packet transaction is just one large atom,
and hence can be executed on \absmachine.  By feeding the same test input to
both the user-supplied packet transaction and the pipelined atom grid that
implements the transaction, we can to verify that the outputs match up
bit-for-bit. We use this to develop an automated tester, \tester , that can
probabilistically verify the correctness of individual compilations in a manner
resembling translation validation~\cite{necula_validation}.

Using \pktlanguage, we express several data-plane algorithms such as flowlet
switching~\cite{flowlet}, data-plane bloom filters~\cite{bloom}, heavy-hitter
detection, CoDel~\cite{codel}, and CONGA~\cite{conga}. We determine if they are
implementable, given specific constraints imposed by \absmachine. In the
process, we provide guidance for how programmable hardware should evolve in the
future to support the needs of data-plane algorithms.
%TODO: I don't quite know what the guidance is yet. Will have it by the weekend.

We distill lessons learned in the process~(\S\ref{s:lessons}) and describe a
design for how \pktlanguage can target P4 as a backend, which would allow
developers to run \pktlanguage on any target that supports P4. We conclude by
placing \pktlanguage in context with related work and by outlining several
areas for future work.

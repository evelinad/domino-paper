\section{Introduction}
\label{s:intro}

%TODO: Consider Alvin's point about global shared state
% being another feature of domino.
Data-plane algorithms~\cite{cestan} are algorithms implemented within a network
switch. These algorithms process every data packet passing through the switch
by transforming the packet and optionally altering the switch's internal state.
Examples include in-network congestion control~\cite{xcp, rcp, pdq, dctcp},
active queue management~\cite{codel}, network measurement~\cite{opensketch,
bitmap_george, elephant_george}, and load-balanced data-plane
routing~\cite{conga}.

Because data-plane algorithms process every packet, an important implementation
requirement is the ability to process packets at line rate.  As a result, these
algorithms are typically implemented using dedicated hardware. However,
hardware designs are rigid and prevent reconfigurability in the field. This
rigidity affects network switch vendors building switch
boxes~\cite{cisco_nexus, dell_force10, arista_7050} based on merchant-silicon
switch chips~\cite{trident, tomahawk, mellanox}, network operators using
merchant silicon within private networks~\cite{google,facebook,vl2}, and
researchers developing new switch algorithms~\cite{xcp, codel, d3, detail,
pdq}.
%
%Today, the only way to implement a new data-plane algorithm at line rate
%is to build hardware for it.

Programmable switching chips~\cite{flexpipe, xpliant, rmt}, which are
competitive with state of the art fixed-function chipsets~\cite{trident,
tomahawk, mellanox}, present an alternative.  These chips allow network
programmers to express their algorithms using primitives provided by the chip.
Programming these chips has become more user-friendly over time.  Initially,
such chips were programmed using proprietary SDKs such as those from
XPliant~\cite{xpliant_sdk, xpliant_sdk2} and Intel~\cite{intel_sdk}. Such SDKs
closely resemble the underlying chipset, making them inconvenient for network
programmers who are more familiar with a language like C.

P4~\cite{p4, p4spec} is a packet-processing language for switching chips that
raises the level of abstraction relative to these SDKs.  For instance, P4
allows the programmer to specify packet parsing and processing without
restricting the set of protocol formats or the set of actions that can be
executed when matching packet headers in a match-action table. Many common
data-plane tasks that require header rewriting can now be expressed without
naturally in P4, such as handling new protocol formats, switching, forwarding,
tunneling, and access control~\cite{dc_p4}.

However, P4 today isn't suited to data-plane algorithms for congestion control,
load balancing, measurement and active queue management. These algorithms don't
rewrite headers, but instead manipulate internal switch state in a manner
unique to each algorithm.  For such data-plane algorithms, network programmers
would prefer the convenience of an imperative language such as C that directly
captures the algorithm's intent without shoehorning them into hardware
constructs such as a sequence of match-action tables like P4 requires them to.
Furthermore, this is predominantly how such data-plane algorithms are
implemented today in sofware routers~\cite{click, dpdk, routebricks}, network
processors~\cite{packetc, nova} and the Linux qdisc subsystem~\cite{qdisc} or
expressed as pseudocode when initially developed~\cite{red, csfq, codel_code,
avq, blue}.

This paper presents \pktlanguage, a new domain-specific language (DSL) for
data-plane algorithms.  \pktlanguage is an imperative language based on C that
allows programmers to express data-plane algorithms using {\em packet
transactions} (\S\ref{s:transactions}).  Packet transactions provide the
abstraction of a sequential block of code that runs to completion on each
packet before executing on the next packet. This is a convenient programming
model, because it allows the programmer to focus on the operations needed for
each packet without worrying about other packets that are concurrently being processed.

% Anirudh->Alvin: I don't like the phrase "to be executed".
% It sounds like a processor, and is a little wierd to me grammatically.
% Technically, we would say to binaries that execute on a family of abs. machines,
% but that would require explaining how we generate the binaries and again gives
% the impression of a processor.
% compiling to \absmachine is my fix to this, and is what we use in the proposal.
We have implemented a \pktlanguage compiler that compiles packet transactions
to a family of abstract machines called \absmachine~(\S\ref{s:absmachine}) (for
Protocol-Independent Switch Architecture). \absmachine generalizes the
Reconfigurable Match-Action Table~\cite{rmt} model and captures essential
features of line-rate programmable switches~\cite{rmt, xpliant, flexpipe}. In
addition, \absmachine introduces {\em atoms} to represent atomic computations
provided natively by a \absmachine machine, much like
load-link/store-conditional, and packed-multiply-and-add on x86 machines
today~\cite{x86_manual}.  Atoms provide hardware support for packet
transactions, similar to how an atomic test-and-set can implement an atomic
increment.

To evaluate \pktlanguage, we express data-plane algorithms~(\S\ref{s:eval})
such as RCP~\cite{rcp}, CoDel~\cite{codel}, heavy-hitter
detection~\cite{opensketch}, and CONGA~\cite{conga}, as packet transactions in
\pktlanguage. Expressing these algorithms involved a straightforward
translation of each algorithm's reference code to \pktlanguage syntax.  The
\pktlanguage compiler guarantees deterministic performance for these
algorithms: all packet transactions that compile run at line rate on a
\absmachine machine, or will be rejected by the compiler.  We use the
\pktlanguage compiler to determine if each algorithm can run at line rate on
several different \absmachine machines that differ in the atoms they provide
(Table~\ref{tab:algos}).

Overall, our results suggest that it is possible to have both a familiar
programming model, resembling DSLs for software routers and NPUs, and line-rate
performance---contrary to concerns raised in~\cite{p4} regarding the
unsuitability of expressive languages for line-rate packet processing.

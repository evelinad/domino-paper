\section{Introduction}
\label{s:intro}

Network switches and routers in modern datacenters, enterprises, and
service-provider networks perform many tasks in addition to standard packet
forwarding. The set of requirements for routers has only increased with time as
network operators seek greater control over performance and security.
Performance and security may be improved using both data-plane and
control-plane mechanisms. This paper focuses on data-plane algorithms. These
algorithms process and transform packets, creating and maintaining state in the
switch. Examples include active queue management~\cite{red,blue,avq,codel,pie},
scheduling~\cite{pifo_hotnets}, congestion control with switch
feedback~\cite{xcp, rcp, pdq, dctcp}, network measurement~\cite{opensketch,
bitmap_george, elephant_george}, security~\cite{dns_change} and load-balanced
routing in the data plane~\cite{conga}.

An important practical requirement for data-plane algorithms is the
ability to process packets at the switch's line rate.  As a result,
these algorithms are typically implemented using dedicated
hardware. Hardware designs are rigid, however, and not reconfigurable
in the field. Thus, to implement and deploy a new algorithm today, the
user must invest in new hardware---a time-consuming and expensive
proposition.

This rigidity affects many stakeholders adversely:
vendors~\cite{cisco_nexus, dell_force10, arista_7050} building network
switches with merchant-silicon chips~\cite{trident, tomahawk,
  mellanox}, network operators deploying
switches~\cite{google,facebook,vl2}, and researchers developing new
switch algorithms~\cite{xcp, codel, d3, detail, pdq}.  To run
data-plane algorithms after a switch has been built, researchers and
companies have attempted to build programmable routers for many years,
starting from efforts on active networks~\cite{active-nets} to network
processors~\cite{npu_survey} to software
routers~\cite{click, dpdk, routebricks}. All these efforts have sacrificed
performance for programmability, typically running an order of
magnitude (or worse) slower than hardware line rates. Unfortunately,
this reduction in performance has meant that these systems are rarely
deployed in production networks, if at all.

Programmable switching chips~\cite{flexpipe, xpliant, rmt, corsa,
  uadp, algo_logic} competitive in performance with state-of-the-art
fixed-function chipsets~\cite{trident, tomahawk, mellanox} are now
becoming available. These chips implement a few low-level hardware
primitives that can be configured by software into a processing
pipeline~\cite{xpliant_sdk,xpliant_sdk2,intel_sdk}.  This approach is
attractive because it does not compromise on data rates and adds
modest area overhead for programmability~\cite{rmt}. From an
operational standpoint, these chips permit field reconfigurability, a
feature that other computational substrates like CPUs, GPUs,
DSPs have benefited from for many years now.

Programmable swithcing chips allow engineers to specify packet parsing and
forwarding without restricting the set of protocol formats or the set
of actions that can be executed when matching packet headers in a
match-action table. Languages such as P4 are emerging as a way to
express such match-action processing in a hardware-independent
way~\cite{p4,p4spec,dc_p4}.

%Data-plane tasks that require large
%amounts of header rewriting can be expressed naturally in
%P4~\cite{dc_p4}.

In contrast to packet header parsing and forwarding, which are stateless
and don't modify state in the data plane, many data-plane algorithms
create and modify algorithmic state in the switch as part of packet
processing. For such algorithms, it is important for programmability to
directly capture the algorithm's intent without requiring it to be
``shoehorned'' into hardware constructs such as a sequence of
match-action tables. Indeed, this is how such algorithms are expressed
in pseudocode~\cite{red, csfq, codel_code, avq, blue}, and implemented
in sofware routers~\cite{click, dpdk, routebricks}, network
processors~\cite{packetc, nova}, and at network
endpoints~\cite{qdisc}.

This paper presents \pktlanguage, a new domain-specific language (DSL)
for data-plane algorithms.  \pktlanguage is an imperative language
with C-like syntax. The key abstraction in \pktlanguage is a {\em
  packet transaction} (\S\ref{s:transactions}): a sequential code
block that is atomic and isolated from other such code blocks. Packet
transactions provide a convenient programming model, because they
allow the programmer to focus on the operations needed for each packet
without worrying about other packets that are concurrently being
processed. To the best of our knowledge, \pktlanguage is the first
high-level imperative language for programming the data plane of
line-rate switches.
%TODO: Do we say high-level language? I think it's useful.

In designing and implementing \pktlanguage, we make three contributions.
\begin{CompactEnumerate}
\item \textbf{A family of machine models called
  \absmachine~(\S\ref{s:absmachine})}.  \absmachine generalizes and abstracts
  essential features of line-rate programmable switches~\cite{rmt, xpliant,
  flexpipe}. \absmachine also models practical constraints limiting stateful
  operations on line-rate switches.  Informed by these constraints, we
  introduce the concept of {\em atoms} to represent a programmable switch's
  instruction set.

\item \textbf{A compiler to compile from packet transactions written in \pktlanguage to
  a \absmachine target~(\S\ref{s:compiler})}. The \pktlanguage compiler
  introduces \textit{all-or-nothing compilation}, where all packet transactions
  accepted by the compiler will run at line rate, or be rejected
  outright---unlike software platforms that have smoother
  performance-complexity tradeoffs.

\item \textbf{An evaluation of \pktlanguage}. We evaluate \pktlanguage's
  expressiveness by writing many data-plane algorithms (Table~\ref{tab:algos})
  in \pktlanguage~(\S\ref{s:eval}), and comparing it with writing data-plane
  algorithms in P4.  Relative to P4, we find that \pktlanguage provides a more
  concise and easier programming model for stateful data-plane algorithms.
  Next, because no existing programmable switch supports the rich set of atoms
  required for our data-plane algorithms, we design a set of compiler targets
  (\S\ref{ss:targets}) based on \absmachine and show that these are feasible in
  a 32 nm standard-cell library with < 10\% chip area overhead.  Finally, we
  compile data-plane algorithms written in \pktlanguage to these targets to
  show how the choice of atoms in a target determines which algorithms it can
  support.
\end{CompactEnumerate}

Overall, we stress two main takeaways. One, it is possible to design
programmable line-rate switches that can express a wide variety of data-plane
algorithms with modest chip area overhead. Second, these switches can be
programmed using a high-level DSL without giving up any performance.

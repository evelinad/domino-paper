\section{Conclusion and Future work}
\label{s:future}
% Multiple transactions
% We don't seem to be discussing types anywhere. We have only ints right now.
% Disclaim that somewhere.

% Can we come up with a hierarchy similar to Herlihy's hierarchy from his
% paper: Wait-Free synchronization?

% Future work:
%
%
% Front-end Optimizations,
% backtracking,
% type systems,
% tables,
% backend: register allocation + placement,
% approximations,
% formally proving correctness, and so on.
% Shared memory.
% I think the most interesting usability problem is providing useful
% diagnostics when things go wrong in the program and the compiler can't
% map the program. Basically, don't want to suffer the same fate as C++ templates.
% What if you could recirculate packets using a loopback interface?

% This snippet below is for when targets support sequenced instructions
%%% If the abstract machine supports more sophisticated instructions that are
%%% equivalent to combinations of two or more three-address instructions, e.g.  the
%%% multiply accumulate instruction (a*b + c), we imagine implementing instruction
%%% selection using tree tiling~\cite{inst_sel}.

\label{s:future}

In this paper we presented \pktlanguage, a new high-level language for expressing 
data-plane
algorithms. While our compiler prototype has achieved promising initial results, 
the design of our language opens up 
a number of interesting open problems to be explored:

\paragraph{Scheduling algorithms}
Packet scheduling~\cite{XXX, XXX} is another important class of data-plane algorithms.
However, they do not currently fit the \pktlanguage packet function abstraction
as scheduling decisions are often based on groups of packets rather than individual
ones. We will extend \pktlanguage, for instance by providing another class of packet
function that takes in a packet buffer as input, for implementing scheduling algorithms.
However, this will raise new challenges in compilation as packet buffers cannot be
easily implemented in our current backend (P4) \ac{check if true}.
\ac{any other classes besides scheduling?}

\paragraph{Richer program constructs}
Likewise, we will extend \pktlanguage to provide additional program constructs
to increase the expressivity of the language. For instance, supporting 
arrays and user-defined types. Such constructs are useful for expressing 
algorithms such as XXX and XXX.

\ac{drop this if not enough space}
\paragraph{Additional backends}
We will investigate compiling \pktlanguage to other backends in addition to P4. 
As a high-level language, \pktlanguage can be targeted to different backends, 
such as XXX, and even directly 
to switch hardware itself. \ac{is there a name for this? I suppose
it's not Verilog or VHDL} The goal is to implement different parts of the source 
program using different backends to leverage the benefits provided by each one:
\ac{give examples}.


\paragraph{Cost Modeling and Systematic Approximation}
As described in Section~\ref{s:context}, 
packet recirculation mechanism can be used to pass stateful variable values  
from a given stage to its predecessor,
in cases where the partitioner fails to combine such
operations into a single stage.
However, doing so results in approximating the semantics of
the input \pktlanguage program, as packets that are already in the pipeline 
while recirculation takes place might be processed based on stale values of the
switch. We will investigate analytical methods to quantify the effects of
such approximation on program semantics by leveraging recent work from
the programming languages community~\cite{sampsonApprox, chisel}. 
Furthermore, we will design cost
models to evaluate different partitionings and incorporate such methods into
the cost model as well. 


\if 0
1. Arrays and other aggregates.
2. Table layout: compiling to P4 and then go down. Make it the P4 compiler's problem.
TCP traffic and end-to-end evaluation of approximation quality like video quality in approximate computing.
So far our approach to evaluating approximations is empirical, not analytical. Analysis would be awesome.
3. Scheduling, by definition, requires looking at multiple packets together to
make a decision on what packet to schedule next, while data-plane algorithms
operate independenly on each packe
\fi

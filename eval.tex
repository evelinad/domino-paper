\section{Evaluation}
\label{s:eval}
% Go through a series of progressively more and more sophisticated models
% and see what we can and can't map.
% The first model:
%  load register into packet temporary for exactly one variable.
%  1 operation on that variable
%  write packet temporary back into register for exactly that one variable.
% The second model:
%  load register into packet temporary for exactly one variable.
%  $N$ operations on that variable.
%  Write packet temporary back into register for exactly that one variable.
% The third model:
%  load register into packet temporary for exactly two variables
%  $N$ operations on that variable.
%  write packet temporary back into register for both variables
% The fourth model:
%  load register into packet temporary for exactly two variables
%  $N$ operations on that variable including conditionals
%  write packet temporary back into register for both variables
% The fifth model:
%  A few blackboxes (such as square root etc) for math functions.
% The sixth model:
% DAG of multiplexers and circuits.
% What about constraints on the number of packet temporaries that can be
% accessed within an atom body? Need to factor those in somehow as well.

% I think we need a neat evocative name for each of these constraint models
% and then we write down templates for them in Sketch.

% Then the bottomline takeaway from the paper is a matrix of algorithms vs.
% stateful atom constraints with an X where it does not map and tick where it
% does. That's the one figure that makes the paper.

% I think for consistency we can use arrays in ALL our examples.


% First, fill this up manually.
% Then see that the compiler generates the same results.
% Then move to generating new results from the compiler

We now express several well-known data-plane algorithms (CONGA~\cite{conga},
RCP~\cite{rcp}, bloom filters~\cite{bloom}, heavy hitter
detection~\cite{opensketch}, determinstic packet sampling) using \pktlanguage
and determine if they are implementable, given a set of atoms provided by
\absmachine. We gradually increase the expressiveness of atoms provided by
\pktlanguage and evaluate whether a specific algorithm is implementable
assuming that atom is provided by \absmachine. Below, we list out the atoms
that we consider in increasing order of expressiveness. We consider only
stateful atoms because converting into three-operand code generates
instructions that map one-to-one to \absmachine for stateless operations.

\subsection{Atoms considered}
%TODO: Mention how atoms are arranged in a hierarchy, each more complex atom
% subsumes the previous one's functionality.
\paragraph{Stateful reads and writes}
This atom allows the program to either read state into a packet temporary or write the
value of a packet temporary or constant into a stateful variable. In SKETCH shorthand,
we represent this atom as:
\begin{figure}
\begin{lstlisting}[style=customc]
pkt.f = x;
\end{lstlisting}
\begin{lstlisting}[style=customc]
x = pkt.f;
\end{lstlisting}
\begin{lstlisting}[style=customc]
x = ??(c);
\end{lstlisting}
\caption{SKETCH shorthand for stateful reads/writes}
\label{fig:read/write}
\end{figure}
%TODO: Can we read an old value and then write a new value atomically?

\paragraph{Increment and decrement}
Next, we consider stateful atoms that allow a particular stateful variable to
be either incremented or decremented by 1 in addition to reading and writing
state variables as above.

\paragraph{Read, modify, write}
We generalize increment and decrement to permit stateful atoms to read state,
add or subtract a constant from it and write it back.

\paragraph{Predicated read, modify, write}
We next add a predication argument that decides whether a state variable is
updated. This atom, in SKETCH shorthand, looks like:
\begin{figure}
\begin{lstlisting}[style=customc]
  if (??(arg1)) x = ??(arg2);
\end{lstlisting}
\begin{eqnarray*}
  ??(arg1) \in \{pkt.field, x\} \\
  ??(arg2) \in \{pkt.field, x\}
\end{eqnarray*}
\end{figure}

\paragraph{An in-order CPU with $N$ instructions}
We generalize predicated read, modify, writes to a generic in-order CPU that
can execute at most $N$ instructions sequentially within an atom. Each of these
instructions are in three-operand code form (\S\ref{s:three_operand}): they can
use the conditional and binary operators, read and write from exactly state
variable and execute at most $N$ instructions.

\paragraph{An in-order CPU that can access two state variables}
This is similar to the one above, except that we permit reads and writes to two
different state variables.

\subsection{Determining if algorithms can be implemented using \absmachine's atoms}

We now consider every combination of atoms listed above, along with all
stateful strongly connected components (TODO: Need better term) in these
algorithms to determine if they are implementable using each atom. We present
our results using the matrix in Table~\ref{fig:eval} that shows whether a
particular algorithm is implementable given a particular set of stateful atoms
available in \absmachine.

\begin{table*}[!t]
  \begin{center}
    \begin{tabular}{|c|c|c|c|c|c|p{2.5cm}|}
  % Other atom templates: R/W, ++/--, ReadAddWrite, Predicated ReadAddWrite, Two-way predication, paired updates
  \hline
    & Read/Write & ++/-- & RMW & Predicated RMW & In-order CPU & In-order CPU, two state vars \\
  \hline
  Deterministic sampling & \xmark & \xmark & \xmark & \cmark & \cmark & \cmark \\
  \hline
  Flowlet switching & \xmark & \xmark & \xmark & \cmark & \cmark & \cmark \\
  \hline
  Heavy hitters & \xmark & \cmark & \cmark & \cmark & \cmark & \cmark \\
  \hline
  Learning bloom filters & \cmark & \cmark & \cmark & \cmark & \cmark & \cmark \\
  \hline %TODO: Need to add non-learning bloom filters as well.
  RCP & \xmark & \xmark & \xmark & \cmark & \cmark & \cmark \\
  \hline
  %%  CONGA & (should do this, two-way state update).
  %%  EWMA & (People are going to freak out about multiply and divide).
  %%  Meter & (too hard)
  %%  XCP & (need to think harder)
  %%  CoDel & (too hard)
  \end{tabular}
\end{center}
\label{table:eval}
\caption{Table summarizing algorithm implementability depending on the atoms provided by \absmachine}
\end{table*}

\section{Evaluation}
\label{s:eval}
% Go through a series of progressively more and more sophisticated models
% and see what we can and can't map.
% The first model:
%  load register into packet temporary for exactly one variable.
%  1 operation on that variable
%  write packet temporary back into register for exactly that one variable.
% The second model:
%  load register into packet temporary for exactly one variable.
%  $N$ operations on that variable.
%  Write packet temporary back into register for exactly that one variable.
% The third model:
%  load register into packet temporary for exactly two variables
%  $N$ operations on that variable.
%  write packet temporary back into register for both variables
% The fourth model:
%  load register into packet temporary for exactly two variables
%  $N$ operations on that variable including conditionals
%  write packet temporary back into register for both variables
% The fifth model:
%  A few blackboxes (such as square root etc) for math functions.
% The sixth model:
% DAG of multiplexers and circuits.
% What about constraints on the number of packet temporaries that can be
% accessed within an atom body? Need to factor those in somehow as well.

% I think we need a neat evocative name for each of these constraint models
% and then we write down templates for them in Sketch.

% Then the bottomline takeaway from the paper is a matrix of algorithms vs.
% stateful atom constraints with an X where it does not map and tick where it
% does. That's the one figure that makes the paper.

% I think for consistency we can use arrays in ALL our examples.

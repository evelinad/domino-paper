\section{Related work}
\label{s:related}

\textbf{Data-Plane Algorithms}
Several data-plane algorithms are now commonplace and available in all
switches. Canonical examples are lookup algorithms based on longest-prefix,
exact, or ternary matches. \pktlanguage focuses on data-plane algorithms that
aren't in wide-spread use and hence aren't widely available because of the
considerable engineering effort required for hardware implementations. Further,
the rapidly growing list of new algorithms~\cite{pdq, d3, detail, codel, conga}
makes it challenging to commit to a hardware implementation for any of them.
Expressing these in \pktlanguage allows network programmers to iterate on these
algorithms at software timescales without committing hardware to any one
algorithm.

\textbf{Abstract machines for line-rate switches}
The protocol-independent switch architecture~\cite{nick_p4} and P4's abstract
switch model~\cite{p4} are two abstract machines for programmable switches.
Relative to P4's abstract switch model, \absmachine contributes the notions of
atoms, sequential execution within atoms, and state encapsulated by atoms.
\absmachine can be seen as an executable instantiation of the
protocol-independent switch architecture~\cite{nick_p4}.

Closest to \absmachine is NetASM~\cite{netasm}, which introduces an
intermediate representation and abstract machine for programmable data planes.
\absmachine differs from NetASM by explicitly targeting line-rate forwarding in
two ways.  First, all state in \absmachine is internal to an atom (and hence
internal to the atom's stage), while NetASM's ATM construct allows access to
shared state from multiple pipeline stages. Second, \absmachine explicitly
enforces constraints on the amount of useful work that can be carried out in an
atom through atom templates.
% --> Their cost models include area and latency. We leave this to P4.

\textbf{Programmable Data Planes}
Several programmable data planes exist today. Software-based solutions such as
Click~\cite{click}, RouteBricks~\cite{routebricks}, and
Fastpass~\cite{fastpass} are flexible but currently lack the performance
required for deployments. Network Processors~\cite{ixp2800, ixp4xx} (NPUs) were
an attempt to bridge the gap.  While NPUs are faster than software routers,
they still remain an order of magnitude slower than merchant silicon
chips~\cite{rmt} and have not found success in the market~\cite{npfail}.

Another approach uses FPGAs to improve over software routers and NPUs; examples
include NetFPGA~\cite{netfpga}, Switchblade~\cite{switchblade},
Chimpp~\cite{chimpp}, and~\cite{silver_bullet}. These designs lag in
performance relative to switching ASICs, and are rarely used today in
production network equipment. The Arista 7124 FX~\cite{7124fx} is a commercial
switch with an on-board FPGA for latency-critical applications, but its
capacity is limited to only 160 Gbits/sec when using the on-board FPGA---an
order of magnitude less than the multi-terabit aggregate capacity of
programmable merchant-silicon chips~\cite{xpliant}. In addition, FPGAs are not
a convenient programming platform.

Relative to these systems, \pktlanguage seeks to provide the best of both
worlds: the line-rate performance of switching ASICs along with a high-level
programming model that is natural to network programmers.

\textbf{Packet-processing languages}
Many systems ease the process of programming the network control plane.
Examples include Frenetic~\cite{frenetic} and Pyretic~\cite{pyretic}. We refer
the reader to~\cite{language_survey} for a survey of such approaches.
In contrast, \pktlanguage focuses on the data plane, not the control plane.

Several DSLs explicitly target data-plane packet processing. Click~\cite{click}
uses C++ to express packet processing for software routers. Imperative
languages such as NOVA~\cite{nova}, packetC~\cite{packetc} Intel's
auto-partitioning C compiler~\cite{intel_uiuc_pldi},PacLang~\cite{paclang_lang,
paclang_partitioner} and Microengine C~\cite{microenginec, intel_ixa} target
network processors such as the Intel IXP~\cite{ixp2800, ixp4xx}.
%PX~\cite{PX}
%is a language that targets packet processing on FPGA platforms.

\pktlanguage's use of C syntax and sequential semantics are inspired by these
languages. However \pktlanguage is more constrained than its predecessors: for
instance, \pktlanguage forbids loops and includes no synchronization constructs
because all state is internal to an atom. These were deliberately chosen to
target line-rate switching chips with deterministic performance and
shared-nothing memory architectures.

As a result, \pktlanguage's compiler presents a different programming model
relative to these prior systems: all \pktlanguage programs that compile run at
line-rate, while those that can't run at line rate are rejected outright. There
is no slippery slope of degrading performance as the program's complexity
increases.

P4~\cite{p4} is an emerging packet-processing language that explicitly targets
line-rate programmable switching chips. However, P4's language constructs
resemble the underlying hardware---requiring the programmer to reason about
concurrency and pipelining.

\textbf{Compiler Techniques}
\pktlanguage's compiler uses three-address codes~\cite{tac}, static-single
assignment form~\cite{ssa}, and if conversion~\cite{if_conversion}, all of
which are well-understood in the compiler literature. However, \pktlanguage's
constrained design allows us to considerably simplify these techniques relative
to their use in mainstream compilers.

The use of strongly connected components to determine atom bodies is based on
similar uses in software pipelining~\cite{software_pipelining} for VLIW and DSP
architectures. However, dependence analysis for loop-carried dependencies in
software pipelining (equivalent to state in \pktlanguage) is more involved than
the simple syntactic analysis of dependencies employed by \pktlanguage. Lastly,
the use of synthesis to fit atom bodies into sketches is based on
Chlorophyll's~\cite{chlorophyll} use of program synthesis for compilation to
unorthodox targets.

\textbf{Hardware compilation}
Prior work ~\cite{nurvadathi, cash, bluespec} has focused on deriving hardware
circuits from high-level imperative descriptions of program behavior for
arbitrary kernels. While these approaches simplify hardware development, the
performance of each kernel depends on its complexity.  \pktlanguage has an
all-or-nothing approach to performance: all code that compiles can run at
line rate or is rejected by the compiler.

Mapping from codelets to atoms is similar to technology mapping~\cite{micheli,
flowmap, spectransform}, where a target hardware circuit (represented as a
graph) is implemented by tiling the graph with several primitive circuits
available in a technology library. However, our problem is simpler because we
only need to verify if a specific codelet maps to a specific atom template; we
don't need to implement a codelet using multiple atoms.

\section{Related work}
\label{s:related}
%TODO: opendataplane.org
%%\textbf{Data-Plane Algorithms}
%%Several data-plane algorithms are now commonplace, e.g., lookup algorithms based
%%on longest-prefix, exact, or ternary matches.  \pktlanguage focuses on
%%data-plane algorithms that aren't widely available because of the engineering
%%effort required for hardware implementations.  Further, the growing list of new
%%algorithms~\cite{pdq, d3, detail, codel, conga} makes it hard to commit to a
%%hardware implementation. \pktlanguage allows network programmers to modify
%%these algorithms more rapidly.

\textbf{Abstract machines for line-rate switches}
%%Relative to P4's abstract switch model~\cite{p4}, \absmachine contributes the
%%notion of atoms.
%%
NetASM~\cite{netasm} is an abstract machine model and intermediate
representation (IR) for programmable data planes that seeks to be portable
across network devices---FPGAs, virtual switches, and line-rate switches.
NetASM's dataflow framework only implements target-independent ``middle-end''
optimizations such as dead-code elimination in its IR. Unlike \pktlanguage,
NetASM neither designs a high-level language, nor implements  a front-end to
compile to NetASM from it (\S\ref{ss:normalization}), nor the back-end to
compile from NetASM to a target (\S\ref{ss:code_gen}). Further, NetASM doesn't
model computational limits in its instruction set and its ATM construct allows
atomic access to shared state from multiple pipeline stages.  Computational
limits are fundamental to providing guaranteed performance at line rate and
sharing state between stages leads to incorrect behavior on line-rate hardware
that doesn't support locks (\S\ref{ss:atoms}).

\textbf{Programmable Data Planes}
Software data planes such as Click~\cite{click},
RouteBricks~\cite{routebricks}, and Fastpass~\cite{fastpass} are flexible but
lack the performance required for large-scale deployments. Network
Processors~\cite{ixp2800, ixp4xx} (NPUs) were an attempt to bridge the gap.
While NPUs are faster than software routers, they remain an order of magnitude
slower than merchant silicon chips~\cite{rmt}. Eden~\cite{eden} provides a
programmable data plane using commodity switches by programming end hosts
alone. \pktlanguage targets programmable switches that increase the scope of
programmable data planes relative to an end-host-only solution. For instance,
they permit us to express in-network congestion control, AQM, and
congestion-aware load balancing (CONGA), which are beyond Eden's capabilities.

An alternative is to use FPGAs; examples include NetFPGA~\cite{netfpga},
Switchblade~\cite{switchblade}, Chimpp~\cite{chimpp}, and~\cite{silver_bullet}.
These designs are slower than switching ASICs, and are rarely used in
production network equipment. The Arista 7124 FX~\cite{7124fx} is a commercial
switch with an on-board FPGA, but its capacity is limited to 160 Gbits/sec when
using the on-board FPGA---~10x less than the multi-terabit capacities of
programmable switch chips~\cite{xpliant}.

Recent work~\cite{lavanya_compiler} focuses on compiling P4 programs to
programmable data planes such as the RMT and FlexPipe architectures. However,
their work focuses on compiling stateless data-plane tasks such as forwarding
and routing, while \pktlanguage focuses on stateful data-plane algorithms.

\textbf{Packet-processing languages}
Many programming languages target the network control plane. Examples include
Frenetic~\cite{frenetic}, Pyretic~\cite{pyretic}, and Maple~\cite{maple}.
\pktlanguage focuses on the data plane. Several DSLs target data-plane packet
processing. Click~\cite{click} uses C++ for packet processing on software
routers. Imperative languages such as NOVA~\cite{nova}, packetC~\cite{packetc},
Intel's auto-partitioning C compiler~\cite{intel_uiuc_pldi},
PacLang~\cite{paclang_lang, paclang_partitioner}, and Microengine
C~\cite{microenginec, intel_ixa} target network processors~\cite{ixp2800,
ixp4xx}.

\pktlanguage's C-like syntax and sequential semantics are inspired by these
DSLs. However \pktlanguage is constrained relative to them: for instance, it
forbids loops and includes no synchronization constructs because there is no
shared state in \absmachine machines. These constraints were chosen to target
line-rate switching chips with shared-nothing memory architectures.
Furthermore, \pktlanguage's compiler presents a different programming model:
all \pktlanguage programs that compile run at line-rate. Unlike an NPU or
software router, there is no slippery slope of degrading performance with
increasing program complexity.

SNAP~\cite{snap} is a recent language for programming stateful data-plane
algorithms using a network transaction: an atomic block of code that treats the
entire network as one big switch~\cite{onebigswitch}. While network
transactions are a higher level of abstraction than packet transactions, they
are limited to data-plane algorithms that can be expressed in a centralized
fashion such as measurement and DDoS prevention. In particular, most
resource-management algorithms in Table~\ref{tab:algos}, e.g. flowlets, RCP,
CONGA, HULL, and AVQ, involve time-varying quantities (queue sizes, packet
arrivals), and are only available as distributed algorithms running on each
switch independently; we don't know of centralized versions for these
algorithms.
%% Keep this description short.
%%SNAP doesn't really get into how stateful computations get implemented on the
%%hardware pipeline of a switch.
%%
%%Say that SNAP can be compiled down to Domino. It doesn't really deal with how
%%individual switch components would get implemented on a hardware pipeline.
%%Domino is an enabler for these kinds of things.

Don't worry about SNAP too much.
%%TODO, should we add this? They seem to be fixing this now ... SNAP doesn't impose
%%constraints on what each switch can do at line rate, and doesn't handle code
%%generation\S\ref{ss:code_gen}. 

%
%P4~\cite{p4} is an emerging packet-processing language that explicitly targets
%line-rate programmable switching chips. While P4 is a natural model for many
%header-manipulation tasks such as switching, ACLs, routing, and
%tunnelling~\cite{dc_p4}, it is ill-suited to programming data-plane algorithms
%that rely on intricate state manipulation.
%
%PX~\cite{PX}
%is a language that targets packet processing on FPGA platforms.

\section{Related work}
\label{s:related}

% If conversion
% http://perso.ens-lyon.fr/christophe.alias/M2/HPL-91-58.pdf
% http://dl.acm.org/citation.cfm?id=567085

% Say why we would still need P4 as the narrow waist for this world.
%%Our compiler takes code written in \pktlanguage and translates it into P4. This
%%source-to-source approach lets us focus on correctness issues when translating
%%\pktlanguage into P4 and lets a P4 compiler handle backend optimizations such
%%as table placement using target-specific information~\cite{lavanya_compiler}.
%%It also allows us to automatically leverage ongoing industry work on P4
%%compilation to hardware targets~\cite{xilinx}.

%P4 itself:
% Mention how we considered compiling down to P4 initially. Tell why it wasn't
% possible because P4 has no constructs for sequential execution, conditional
% operators, or arithmetic expressions and how we have now proposed this for
% inclusion into the standard based on this work.  Maybe mention the
% p4-semantics repo to show our proposal Either say it's under consideration at
% P4.org or has been standardized already.

%TODO: Distinguish against Shahbaz and NickF's NetASM paper that introduces atomic
% in the related work.
% The NetASM model does quite a bit of this.
% I think the differences from Banzai are:
% --> They don't constrain the complexity of atoms.
% --> They allow atomic access to state across multiple stages.
%     I don't know how this guarantees correctness.
% --> We restrict atoms to be within one stage.
% --> We aren't defining an IR at all. We allow arbitrary code within atoms.
% --> Their cost models include area and latency. We leave this to P4.
% --> We don't target the same breadth of devices that they do.
% I should talk to Shahbaz about this.

% microenginec, intel ixpc and everything else from the packetC paper.

We divide related work into three categories: data-plane algorithms, programmable
data planes, and languages for packet processing.

\subsection{Data-Plane Algorithms} 
As defined earlier, data-plane algorithms are those algorithms that run within
a switch and execute on every packet. Some data-plane algorithms are now
commonplace and available in all switches: lookup algorithms based on
longest-prefix match, exact match, ternary match and range-search are probably
the best known of these.

  Our focus on data-plane algorithms that aren't widely available on all
switches today because of the considerable engineering effort required for
hardware implementations and the rapidly changing nature of these algorithms.
For instance, load-balancing algorithms such as CONGA~\cite{conga} are
available only on a specific line of CISCO switches.  Active queue-management
algorithms such as CoDel~\cite{codel} need to be extensively
modified~\cite{pie} to suit the architecture of a hardware switch.  Explicit
congestion-control algorithms such as RCP~\cite{rcp} and XCP~\cite{xcp}, and
measurement algorithms such as OpenSketch~\cite{opensketch} have been evaluated
only in FPGA-based platforms.  The ever-growing list of new
algorithms~\cite{pdq, d3, detail} makes it challenging to commit to a hardware
implementation of any of them.

\subsection{Programmable Data Planes}
Several proposals for programmable data planes exist today. Software-based
solutions such as ~\cite{click, fastpass, flexplane} are extremely flexible but
have trouble scaling. NPUs ~\cite{intel} were an attempt to bridge the gap.
While NPUs are faster than pure software routers, they still remain an order of
magnitude slower than merchant silicon chips~\cite{rmt}. Our focus is on
programmable switch architecture that run at line rate: i.e. with performance
comparable to off the shelp merchant silicon chips. These are the most
performant of the programmable switch options and the least expressive, making
programming these devices especially challenging.

\subsection{Packet-processing languages}
Several DSLs explicitly target packet processing. P4~\cite{p4} is the backend
we use in this chapter and is used to program chipsets based on the
RMT~\cite{rmt} and Intel FlexPipe~\cite{flexpipe} architecture. PacketC is a
more expressive packet-processing language that allows memory sharing using
locks and synchronization because the underlying hardware that it targets
allows the same.

\pktlanguage{} is complementary to these languages that are close to the
hardware. For instance, by translating into P4, \pktlanguage{} can automatically be
run on all targets that P4 supports~\cite{lavanya_compiler}. Like any other
nascent language, P4 is likely as more experience is accumulated. In light of
this, our work can be viewed as a recommendation to extend P4 with
transactional actions that span multiple tables, and can benefit from the rest
of P4's abstractions: arbitrary match entries, typed header fields, control
flow across tables, and programmable packet parsing.

\subsection{Transactions}
Transactions are a staple of modern databases. However, they are also
ubiquitous in the world of networking: even if they go by multiple names. For
instance, Click's element abstraction~\cite{click} allows code to be run
without being pre-empted, Intel's compiler for the IXP
architecture~\cite{intel} featured a similar high-level transactional
specification called the PPE for packet processing enginer.
Flexplane~\cite{flexplane} also allows the user to specify code in a high-level
language such as C++ in a similar transactional style and Maple~\cite{maple}
applies the notion on an algorithmic policy for SDN controllers: in essence,
a network-wide transaction.

 \pktlanguage{} is inspired by these prior systems in its use of transactions.
However, it's emphasis on compiling these transactions to extremely minimal,
constrained architectures like the RMT model is new. These architectures also
result in a new set of tradeoffs: a larger program doesn't run slowly. Instead,
it doesn't run at all, unless suitably approximated.

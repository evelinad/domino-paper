\subsection*{Abstract}
Data-plane algorithms execute on every packet traversing a network
switch; they encompass many schemes for congestion control, network
measurement, active queue management, and load balancing. Because
these algorithms are implemented in hardware today, they cannot be
changed after being built. To address this problem, recent work has
proposed designs for programmable line-rate switches. However, these
designs don't yet support many network resource management schemes,
and the languages to program them closely resemble the underlying
hardware, rendering them inconvenient for this purpose.

% Anirudh->Alvin: I don't like the phrase "to be executed".
% It sounds like a processor, and is a little wierd to me grammatically.
% Technically, we would say to binaries that execute on a family of abs. machines,
% but that would require explaining how we generate the binaries and again gives
% the impression of a processor.
% compiling to \absmachine is my fix to this, and is what we use in the proposal.
This paper presents \pktlanguage, a C-like imperative language for data-plane
algorithms. \pktlanguage introduces the notion of a {\em packet
transaction}, defined as a sequential code block that is atomic and
isolated from other such code blocks.
%Programmers write data-plane algorithms in \pktlanguage using
%packet transactions: sequential code blocks that run to completion on each
%packet before processing the next one.  The \pktlanguage compiler
compiles \pktlanguage code to \absmachine, a family of abstract
machines that map to the architecture of emerging programmable switch
chipsets. We show how \pktlanguage enables several data-plane
algorithms written in C syntax to run at hardware line rates.

% based on
%emerging programmable line-rate switches.  We evaluate \pktlanguage by
%expressing several data-plane algorithms in \pktlanguage and compiling them to
%different \absmachine machines.

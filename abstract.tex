\subsection*{Abstract}
Data-plane algorithms are algorithms that execute on every packet. They
encompass many packet-processing routines including congestion control, network
measurement, active queue management, and load balancing. Such algorithms are
implemented in hardware today, prohibiting reconfigurability in the field.
While programmable line-rate switches have been proposed, they are difficult to
utilize due to the low-level nature of the languages used to program them.

This paper presents \pktlanguage, a high-level\ac{I thought we are not doing the 
high level language story anymore}, imperative language for
expressing data-plane algorithms. Programmers write data-plane algorithms in
\pktlanguage using packet transactions, which conceptually are 
sequential code blocks that run to
completion on each packet before processing the next one. The \pktlanguage compiler translates
the code to be executed on \absmachine, a family of abstract machines
based on emerging programmable line-rate switch architectures. We evaluate
\pktlanguage by expressing several data-plane algorithms in \pktlanguage and
compiling them to different \absmachine machines. Our results suggest---contrary
to concerns raised in~\cite{p4}---that there is no reason to sacrifice the
convenience of high-level programming for line-rate performance.\ac{If we are not
going for high level language then we shouldn't be drawing this conclusion}

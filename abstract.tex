\begin{abstract}

Many algorithms for congestion control, scheduling, network measurement, active
queue management, and traffic engineering require custom processing of packets
as they traverse the data plane of a network switch. To run at line rate, these
data-plane algorithms must be implemented in hardware. With today's switch
hardware, algorithms cannot be changed, nor new algorithms installed, after a
switch has been built.

This paper shows how to program data-plane algorithms in a high-level language
and compile those programs into low-level microcode that can run on emerging
programmable line-rate switching chips. The key challenge is that many
data-plane algorithms create and modify algorithmic state. To achieve line-rate
programmability for stateful algorithms, we introduce the notion of a {\em
packet transaction}: a sequential packet-processing code block that is atomic
and isolated from other such code blocks.

We have developed this idea in \pktlanguage, a C-like imperative language to
express data-plane algorithms. We show with many examples that \pktlanguage
provides a convenient way to express sophisticated data-plane algorithms, and
show that these algorithms can be run at line rate with modest estimated
chip-area overhead.

\end{abstract}

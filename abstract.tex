\subsection*{Abstract}
Data-plane algorithms are algorithms that execute on every packet and encompass
many packet-processing routines including congestion control, network
measurement, active queue management, and load balancing. Such algorithms are
implemented in hardware today, prohibiting reconfigurability in the field.
Programmable line-rate switches have been proposed. However, the languages that
program them still resemble the underlying hardware and are inconvenient to use.

% Anirudh->Alvin: I don't like the phrase "to be executed".
% It sounds like a processor, and is a little wierd to me grammatically.
% Technically, we would say to binaries that execute on a family of abs. machines,
% but that would require explaining how we generate the binaries and again gives
% the impression of a processor.
% compiling to \absmachine is my fix to this, and is what we use in the proposal.
This paper presents \pktlanguage, an imperative language for data-plane
algorithms. Programmers write data-plane algorithms in \pktlanguage using
packet transactions: sequential code blocks that run to completion on each
packet before processing the next one. The \pktlanguage compiler compiles
\pktlanguage code to \absmachine, a family of abstract machines based on
emerging programmable line-rate switches.  We evaluate \pktlanguage by
expressing several data-plane algorithms in \pktlanguage and compiling them to
different \absmachine machines.

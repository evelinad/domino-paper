\subsection*{Abstract}
% What did you build?
% What did you learn?
% I think what we learned is that we can have both the convenience of
% high-level code and the performance of line-rate switches.
Data-plane algorithms are algorithms that execute on every packet and encompass
a wide variety of packet-processing routines including congestion control,
network measurement, active queue management, and load balancing. Such 
algorithms are implemented in hardware today, whose inflexibility prohibits
experiments with new designs. While programmable switches have been proposed, 
they are difficult to utilize due to the low-level nature of the languages
used to program them.

This paper presents \pktlanguage, a new language for expressing data-plane
algorithms on programmable switches. \pktlanguage is a high-level, imperative
language where developers express data-plane algorithms using packet
transactions: sequential blocks of code that run to completion on each packet
before processing the next one. We have implemented a compiler that translates
\pktlanguage programs into pipelined implementations that run on \absmachine
(for Protocol-Independent Switch Architecture), a family of abstract machines
modeled after emerging programmable switch architectures. We evaluate
\pktlanguage by expressing several well-known data-plane algorithms and
compiling them to different \absmachine machines.

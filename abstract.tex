\subsection*{Abstract}
Data-plane algorithms execute on every packet traversing a network switch; they
encompass many schemes for congestion control, scheduling, network measurement,
active-queue management, security, and load balancing. Because these algorithms
are implemented in hardware today, they cannot be changed after being built. To
address this problem, recent work has proposed designs for programmable
line-rate switches.  However, these chips have only been used to program
stateless data-plane tasks, such as packet forwarding and access control. By
contrast, many data-plane algorithms create and modify algorithmic state on a
as part of their packet processing.

This paper presents \pktlanguage, a C-like imperative language to express
data-plane algorithms. \pktlanguage introduces the notion of a {\em packet
transaction}, defined as a sequential code block that is atomic and isolated
from other such code blocks.  The \pktlanguage compiler compiles \pktlanguage
code to \absmachine, a family of machine models based on emerging
programmable switch chipsets. We evaluate \pktlanguage by first designing
concrete \absmachine machines that support a variety of data-plane algorithms
with modest die area overhead. We then show how \pktlanguage simplifies
programming them, relative to current languages for programmable switches.

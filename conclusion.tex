\section{Conclusion}
\label{s:conclusion}

This paper presented \pktlanguage, an imperative language that allows
programmers to write packet-processing code using packet transactions:
sequential code blocks that run to completion on every packet before processing
the next one. The \pktlanguage compiler compiles packet transactions
to \absmachine, a family of abstract machines based on
programmable line-rate switch architectures~\cite{flexpipe, xpliant, rmt}. Our
results suggest that it is possible to have both a familiar programming model
and line-rate performance---if the algorithm can indeed run at line rate.

%TODO: Don't know if this serves the right purpose, but this is genuinely
% what I think about the work. That the compiler is rather straightforward ...
This is somewhat in contast to claims in~\cite{p4} that a high-level language
such as Click~\cite{click} is unsuitable for line-rate switches. \pktlanguage
shows that---with careful design---it is possible to find a sweet spot between
expressiveness and performance, somewhere between P4 today and Click. Further,
as \S\ref{s:compiler} shows, careful language design simplifies the compiler.
In summary, packet transactions are more familiar to the user, do not sacrifice
performance, and result in a simple compiler.  Packet-processing languages such
as P4 are still in their infancy and we hope these results prompt a larger
conversation on the benefits of user-friendly abstractions, such as packet
transactions, for packet processing.

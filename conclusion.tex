\section{Conclusion}
\label{s:conclusion}

This paper presented \pktlanguage, a high-level language that allows
programmers to write packet-processing code using packet transactions:
sequential code blocks that run to completion on every packet before processing
the next packet. The \pktlanguage compiler compiles packet transactions in
\pktlanguage to an instance of \absmachine, a family of abstract machines
modeled after emerging programmable line-rate switch
architectures~\cite{flexpipe, xpliant, rmt}. The compiler guarantees that all
packet transactions that compile will run at line rate and rejects \pktlanguage
programs that cannot run at line rate (Table~\ref{table:eval}). Our results
suggest that it is possible to simultaneously have the convenience of
high-level programming and the performance of line-rate switches. While much
work (\S\ref{s:limitations}) remains before packet transactions can run on real
hardware, we hope to have convinced the reader that it is possible and worth
exploring further as a simple model for line-rate data-plane programming.

\section{Conclusion}
\label{s:conclusion}

This paper presented \pktlanguage, a high-level language that allows
programmers to write packet-processing code using packet transactions:
sequential code blocks that run to completion on every packet before processing
the next one. The \pktlanguage compiler compiles packet transactions in
\pktlanguage to \absmachine, a family of abstract machines modeled after
emerging programmable line-rate switch architectures~\cite{flexpipe, xpliant,
rmt}. Our results suggest that it is possible to simultaneously have the
convenience of high-level programming and the performance of line-rate
switches, contrary to claims in~\cite{p4} that a high-level language is
insufficiently constrained for programming line-rate switches.

In light of those claims, our position is that---when carefully
constrained---it is possible to design a language that finds a sweet spot
between expressiveness and performance somewhere between the extremes of P4 as
it stands today and Click. P4 as a language is still evolving and we hope some
of these results prompt a larger conversation around the utility of having
higher-level abstractions (such as transactions) in P4. As this paper shows,
these abstractions come at no cost to performance. While much work remains
before packet transactions can run on real hardware, we hope to have convinced
the reader that it is possible.

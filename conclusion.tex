\section{Conclusion}
\label{s:conclusion}

This paper presented \pktlanguage, a high-level language that allows
programmers to write packet-processing code using packet transactions:
sequential code blocks that run to completion on every packet before processing
the next one. The \pktlanguage compiler compiles packet transactions in
\pktlanguage to \absmachine, a family of abstract machines based on
programmable line-rate switch architectures~\cite{flexpipe, xpliant, rmt}. Our
results suggest that it is possible to have both the convenience of high-level
programming and the performance of line-rate switches, contrary to claims
in~\cite{p4} that a high-level language is too expressive to program line-rate
switches.

In response to ~\cite{p4}'s claims, we believe that---with careful design---it
is possible to find a sweet spot between expressiveness and performance,
somewhere between the extremes of P4 today and Click. Further, as \pktlanguage
shows, careful language design pays off when implementing the compiler for the
langauge. P4 as a language is still evolving and we hope these results prompt a
larger conversation on the benefits of higher-level abstractions, such as
packet transactions, in P4. As this paper shows, these abstractions do not
sacrifice performance. While work remains (\S\ref{ss:real}) before packet
transactions can run on switches, we hope to have convinced the reader that it
will soon be possible.

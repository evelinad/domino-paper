\section{Conclusion}
\label{s:conclusion}

This paper presented \pktlanguage, a C-like imperative language that allows
programmers to write packet-processing code using packet transactions:
sequential code blocks that are atomic and isolated from other such code
blocks. The \pktlanguage compiler compiles packet transactions to \absmachine,
a machine model based on programmable line-rate switch
architectures~\cite{flexpipe, xpliant, rmt}. Our results suggest that it is
possible to have both a familiar programming model and line-rate
performance---i.e. if the algorithm can indeed run at line rate.
Packet-processing languages are still in their infancy; we hope these results
prompt further work on programming abstractions for packet-processing hardware.

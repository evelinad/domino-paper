\section{Conclusion}
\label{s:conclusion}

This paper presented \pktlanguage, a high-level imperative language that allows
programmers to write packet processing code using packet transactions:
sequential code blocks that conceptually run to completion on every packet
before processing the next packet. The \pktlanguage compiler compiles packet
transactions in \pktlanguage to an instance of \absmachine, a family of
abstract machines modeled after emerging programmable line-rate switch
architectures~\cite{intel, xpliant, rmt}. The compiler guarantees that all
packet transactions that compile will run at line rate and rejects \pktlanguage
programs that cannot be run at line rate (\S\ref{s:eval}).

Our results suggest that it is possible to have the best of both worlds: the
convenience of high-level programming and the performance of line-rate
switches, provided we pay careful attention to the design of the language
itself. As our paper shows, constraining the language also has the pleasant
side effect that it simplifies the compiler. That said, we view this only as a
first step and much work (\S\ref{s:limitations}) remains before we can run
transactional programs on actual programmable switches. We hope this paper has
convinced the reader that this is possible and an approach that is worth
refining further.

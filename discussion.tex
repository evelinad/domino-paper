\section{Discussion}
Packet transactions provide a pathway to take algorithms that were hitherto
meant only for software routers and run them on emerging programmble line-rate
switching chips. However, more work must be done before packet transactions
are ready for production use.

\begin{CompactEnumerate}
\item With packet transactions, we strove for the strongest and simplest
semantics possible: transactional semantics that provide the notion of an
atomic and isolated block of code. Although these semantics make it easier to
reason about performance, they exclude algorithms that cannot be run exactly at
line rate. Are weaker semantics sensible? One possibility is approximating
transactional semantics by executing the code on a subset of the packet stream.
This provides an increased time budget for each packet, potentially allowing
the packet to be {\em recirculated} through the pipeline multiple times
for packet processing. Another possibility is for the implementation to
be eventually consistent with the transactional model, perhaps by updating
state over many clock cycles and guaranteeing that it will be correct only if
there are no subsequent state updates.
\item Our current implementation doesn't handle multiple transactions and the
associated issues of composing transactions. A policy language to
specify multiple transactions executing on different, possibly overlapping,
subsets of packets is an area for future work.
%% \item Our compiler doesn't provide completeness: the guarantee that for any
%% algorithm, if there exists any way to map an algorithm to a line-rate switch,
%% it will be found. We haven't required this for the examples in our evaluation
%% because the compiler does find a solution if one exists. However, as we scale
%% to larger code examples, having this guarantee would be useful.
% This is confusing. The writing makes it sound like SKETCH is complete, but the compiler isn't.
% Removing for now.
\item Our compiler doesn't aggressively optimize. For instance, it may be
possible to fuse two stateful codelets that independently increment two
separate counters into the same instance of the Pairs atom. However, by
carrying out a one-to-one mapping from codelets to the atoms implementing them,
our compiler precludes these optimizations.  Developing an {\em optimizing}
compiler for packet transactions is an area for future work.
\item Finally, our design process for atoms is largely manual.  Formalizing
this design process and automating it into an atom-design tool would be useful
for switch designers. For instance, given a corpus of data-plane algorithms,
can we automatically mine this corpus for stateful and stateless codelets, and
design an atom (or atoms) that captures the computations required by some (or
all) of them?

%%\ac{would interleaving stmts from multiple transactions be a potential
%%optimization here (as in db transactions)?}
%% Possibly: I don't quite know how to reason about the semantics yet :) We should talk about this, sometime!

\end{CompactEnumerate}

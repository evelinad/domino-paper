\section{Discussion}
\new{
Packet transactions provide a pathway to take algorithms that were hitherto
meant only for software routers and run them on emerging programmable line-rate
switching chips. However, more work must be done before packet transactions
are ready for production use.
}

\new{
\begin{CompactEnumerate}
\item With packet transactions, we strove for the strongest and simplest
semantics possible: transactional semantics that provide the notion of an
atomic and isolated block of code. Although these semantics make it easier to
reason about performance, they exclude algorithms that cannot run exactly at
line rate. Are weaker semantics sensible? One possibility is approximating
transactional semantics by only processing a sampled packet stream.
This provides an increased time budget for each packet, potentially allowing
the packet to be {\em recirculated} through the pipeline multiple times
for packet processing.
\item Our compiler doesn't aggressively optimize. For instance, it is
possible to fuse two stateful codelets incrementing two
independent counters into the same instance of the Pairs atom. However, by
carrying out a one-to-one mapping from codelets to the atoms implementing them,
our compiler precludes these optimizations.  Developing an {\em optimizing}
compiler for packet transactions is an area for future work.
\item Supporting multiple packet transactions in \pktlanguage also requires
further work. In particular, when a switch executes multiple transactions,
there may be opportunities for whole-program compilation~\cite{whole_program}, which
goes beyond compiling individual transactions. For
instance, the compiler could detect computations common to multiple
transactions and execute them only once.
\item Finally, we have a manual design process for atoms.  Formalizing
this design process and automating it into an atom-design tool would be useful
for switch designers. For instance, given a corpus of data-plane algorithms,
can we automatically mine this corpus for stateful and stateless codelets, and
design an atom (or atoms) that captures the computations required by some (or
all) of them?
\end{CompactEnumerate}
}

%It could also interleave statements
%between transactions provided the interleaving doesn't violate the semantics of
%composing transactions (\S\ref{ss:multiple}).

%Another possibility is for the implementation to
%be eventually consistent with the transactional model, perhaps by updating
%state over many clock cycles and guaranteeing that it will be correct only if
%there are no subsequent state updates.

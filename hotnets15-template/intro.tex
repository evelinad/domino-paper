\section{Introduction}
\label{s:intro}

Data-plane algorithms~\cite{cestan} are algorithms that are implemented within
a network switch. These algorithms process every data packet that passes
through the switch, transforming the packet and/or state of the switch itself.
Examples of such algorithms including networking staples such as prefix
lookup~\cite{lookup} and deep packet inspection~\cite{ids}, algorithms for
in-network load balancing~\cite{conga, ecmp}, congestion-control that uses
feedback from switches~\cite{xcp, rcp, pdq, dctcp}, and network
measurement~\cite{minsketch, heavy_hitters}.

Because data-plane algorithms process every packet going through the switch, a
major performance requirement for data-plane algorithms is the ability to run
at the switch's line rate. As a result, these algorithms are primarily implemented
using dedicated hardware. Hardware designs are rigid and inflexible, making it difficult
for network operators and researchers to experiment with new algorithms.

As such, programmable switch architectures~\cite{flexpipe, xpliant, rmt} allow
designers to express their algorithms using a domain-specific language (DSL)
(e.g., P4~\cite{p4} and Protocol-Oblivious Forwarding~\cite{pof}) and present
a compelling alternative.  Unfortunately, these DSLs are too close to the
underlying network hardware.

%%For instance, P4 mixes sequential semantics and
%%parallel semantics. making them difficult for designers to work with
%%as designers need to make explicit the control flow within different parts of
%%their code that is typically implicit in high-level imperative languages.

Besides, a large body of packet-processing code is already written in a
high-level language like C for platforms such as software routers~\cite{click},
the Linux qdisc subsystem~\cite{qdisc} and Network Processors~\cite{npu}.

In this paper, we propose a new DSL, \pktlanguage, for expressing 
data-plane algorithms. \pktlanguage
is a high-level, imperative language that allow 
designers express data-plane algorithms as transactions~\cite{transactions}, 
i.e., blocks of code that conceptually run to completion on a packet without any
interruptions \ac{is this the formal definition of transactions?}. Using 
transactions allow designers to easily express a wide variety of algorithms,
as have been demonstrated in implementing packet-processing code for 
other network elements such as software routers and NPUs~\cite{click, intel, qdisc}.

We have built a compiler for \pktlanguage~(\S\ref{s:compiler}), 
which automatically partitions
transactional code from the designer into multi-stage pipelines to be implemented
on programmable switches. In the following, we first describe the low-level
hardware pipeline exposed by programmable switches today~(\S\ref{s:architecture}),
followed by a description of \pktlanguage~(\S\ref{s:language}). We then
explain our compiler toolchain~(\S\ref{s:compiler}, \S\ref{s:advanced}), 
along with preliminary
experiments in implementing real-world data-plane algorithms using \pktlanguage
~(\S\ref{s:evaluation}). We then 
conclude by outlining several areas of future work~(\S\ref{s:conclusion}).


% old text below
\if 0
Data-plane algorithms~\cite{cestan} are algorithms running within a switch that
operate on every data packet, either transforming the packet or some state on
the switch itself. Examples of such algorithms include prefix lookup, deep
packet inspection, measurement algorithms~\cite{minsketch}, load-balancing
algorithms~\cite{conga}, and in-network congestion-control algorithms.

Today, switch designers go to painstaking lengths to engineer hardware
implementations for a chosen few of these algorithms. Because these algorithms
are carefully curated by a switch designer for inclusion into a hardware
switch, most data-plane algorithms invariably don't find their way into a
switch. At the same time, there is no dearth of academic proposals for new
data-plane algorithms~\cite{minsketch, codel, pdq}.

Emerging programmable switch architectures~\cite{flexpipe, xpliant, rmt} allow
line-rate packet processing to be expressed as a program in a domain-specific
programming language~\cite{p4}.  This program specifies packet processing as a
sequence of match-action tables where both the match and action components are
highly flexible. The match key can be any combination of header fields
including user-defined header fields, while the action is built up from a small
core set of programmable packet processing primitives.

With the advent of programmable switch architectures, this paper asks if
data-plane algorithms can be implemented as programs on programmable line-rate
switches. Answering this question positively would allow us to view data-plane
algorithms as software, resulting in much faster iteration between algorithms
than was previously possible.

We begin this paper by surveying several data-plane
algorithms~(\S\ref{s:dataplane}) for shared characteristics. Based on these
observations, we propose a language for data-plane
algorithms~(\S\ref{s:language}) that expresses data-plane algorithms as
transactions~\cite{transactions}: blocks of code that run to completion without
any interruptions. Our use of transactions is inspired by the widespread use of
the transactional style in expressing packet-processing code in software
routers and NPUs~\cite{click, intel, qdisc}.

While transactions are a natural way for programmers to express code, there is
a large semantic gap between the high-level language of transactions and the
low-level programmable pipeline~(\S\ref{s:architecture} exposed by programmable
switches today. We bridge this gap by building a compiler ~(\S\ref{s:compiler})
that automatically pipelines transactional code. In many cases, the compiler
can take advantage of the approximate nature of many data-plane algorithms
~(\S\ref{ss:advanced}). We evaluate~(\S\ref{s:evaluation}) the pipelined code
and conclude by outlining several areas of future work~(\S\ref{s:conclusion}).

% TODO: Reiterate contributions if we need to. Most HotNets papers don't do this.

\fi

\section{Introduction}
\label{s:intro}
Data-plane algorithms today are hand picked carefully by switch designers for
inclusion into a new hardware-based merchant-silicon switch.

Programmable switch architectures could probably change that.
(Programmability is driven by the realization that it doesn't cost
too much. The area is already dominated by memories and Serial Link IO., so
programmability adds only about 15\% net-net)

Now, looking at data-plane algorithms. We observe that these are best expressed as transactions
Which looks very different from the architecture of these programmable switches (a pipeline)
The ideal place to bridge this gap is a compiler.

Contributions:
1. A way to express data-plane algorithms as transactions, while considerably restricting the code within the transaction to make them feasible on line-rate switches.
2. Compiler to translate it into programmable hardware.
3. How this compiler world-view help us longer-term against a landscape of currently evolving switch architectures and perenially evolving algorithms.

\section{Data-plane algorithms}
\label{s:dataplane}
Describe a few examples (CONGA, CoDel, RED, XCP, min. sketch, Bloom Filters)

\subsection{Distinguishing characteristics}
1. too expressive for P4.
2. Intricate control flow.
3. Best structured as transactions. Inspired by the use of transactions to structure other kinds of network programs: Click uses it for everything,
Maple uses it for control programs,
The Intel paper uses it for NPU programs.
The Linux qdisc system uses it as well.

I think, this way, we directly defend against criticism that transactions are an old idea, even for networking. All we are claiming is that transactions are the right way to structure data-plane algorithms---even those running at line rate. Our contribution relative to these prior systems is that we show how transactions can be compiled down to line-rate switches.


\subsection{Approximate nature of some algorithms}
1. Some can also be approximated (RED, PIE, sketches, no spec).
2. In fact, already happens today (RED @ CISCO is different from the RED spec), and PIE is CISCO's simplified version of CoDel.
3. Briefly mention, how this is useful to leverage.

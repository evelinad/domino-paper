\section{Limitations and Future Work}
\label{s:limitations}

\subsection{Limitations}
While \pktlanguage's compiler guarantees correctness, it doesn't attempt
optimality.  Specifically, \pktlanguage doesn't reduce or consolidate the
number of packet temporary variables it creates. By flattening all statements
to three-operand code form, it doesn't take advantage of more powerful atoms
that might be available in the \absmachine instance.

Taken together, this increases the width (because there might be redundant
atoms that compute the same quantity twice in the same stage) and depth of the
resulting pipeline (expression flattening could cause an additional stage,
where one stage would have sufficed). We plan to address this by incorporating
both standard compiler optimizations such as common subexpression elimination,
copy propagation, and dead code elimination, and techniques such as
superoptimization~\cite{stoke, superoptimizer} that explicitly aim for optimal
code.

We also currently assume there is only one packet transaction that needs to be
implemented by \pktlanguage. Extending \pktlanguage to handle multiple
potentially interacting packet transactions is another area of future work.

% Anirudh->Hari: Do you buy this?
Lastly, \pktlanguage doesn't currently generate an implementation that runs on
a programmable switch. We initially considered compiling \pktlanguage to
P4, but decided against it because P4 currently has no constructs for
sequential execution within a pipeline stage. Sequential execution within a
stage is required to execute the codelets/atoms produced by \pktlanguage and
guarantee correctness. Based on this work, we have submitted a proposal for
sequential execution~\cite{p4-semantics} to the P4 language consortium. In the
future, if P4 supports sequential execution, \pktlanguage could target P4 as a
backend, allowing us to leverage ongoing work ~\cite{netronome,
xilinx,lavanya_compiler} in compiling P4 to hardware targets.

\subsection{Future Work}
Currently, \pktlanguage rejects programs that don't compile with an error
message.  In the future, we plan to improve upon this by automatically
suggesting ways in which the user could approximate the program so that it
could compile. We imagine doing this in two ways. First, \pktlanguage could
suggest ways to simplify a codelet that doesn't map so that it does map.
Second, we could use primitives such as P4's recirculate primitive~\cite{p4spec}
to send a packet back into the switch pipeline if the packet
requires more processing than is available in one traversal of the switch
pipeline. Reasoning about the semantics of these
approximations~\cite{sampsonApprox, chisel} is also another area for future
work.

We also plan to improve the \pktlanguage language by providing a stronger type
system, type inference, user-defined functions and types, and language
constructs for packet processing such as match-action tables. Another area of
future work is looking at whether \pktlanguage can be used (or extended) to
express data-plane algorithms such as scheduling, active queue management, and
buffer-management schemes such as Broadcom's dynamic buffer threshold policy.


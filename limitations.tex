\section{Limitations and Future Work}
\label{s:limitations}

\textbf{Limitations}
While \pktlanguage's compiler guarantees correctness, it doesn't guarantee
optimality.  The \pktlanguage compiler doesn't reduce or consolidate the number
of temporaries it creates. By flattening all statements to three-operand code
form, it doesn't take advantage of more powerful atoms provided by the
\absmachine machine. By mapping stateful codelets to stateful atoms
one-to-one, it doesn't take advantage of more powerful stateful atoms, such as
the paired update atom, which could express the operations of two stateful
codelets in one atom.

\textbf{Future Work}
Currently, \pktlanguage rejects programs that don't compile with an error
message.  In the future, we plan to improve upon this by automatically
suggesting ways in which the user could approximate the program so that it
could compile. We imagine doing this in two ways. First, \pktlanguage could
suggest ways to simplify a codelet that doesn't map so that it does map.
Second, we could use primitives such as P4's recirculate
primitive~\cite{p4spec} to send a packet back into the switch pipeline if the
packet requires more processing than is available in one traversal of the
switch pipeline. Reasoning about the semantics of these
approximations~\cite{sampsonApprox, chisel} is also another area for future
work.

We also plan to enhance the \pktlanguage language by providing a stronger type
system, type inference, user-defined functions and types, and language
constructs for match-action tables. Another area of future work is looking at
whether \pktlanguage can be used (or extended) to express data-plane algorithms
such as scheduling, active queue management, and buffer-management
schemes~\cite{broadcom_buffer}.
